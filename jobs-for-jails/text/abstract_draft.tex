\documentclass[12pt]{article}
\usepackage[margin=1in]{geometry}
\usepackage{setspace}
\usepackage{natbib}
\usepackage{hyperref}

\title{\textbf{Jobs for Jails? Economic Self-Interest and Public Support for Immigration Enforcement}}

\author{
Charles Crabtree\thanks{Senior Lecturer, School of Social Sciences, Monash University and K-Club Professor, University College, Korea University. Email: charles.crabtree@monash.edu}
}

\date{\today}

\begin{document}

\maketitle

\begin{abstract}
\noindent What factors shape public support for immigration enforcement in local communities? While existing research emphasizes the roles of ideology, partisanship, and racial attitudes in structuring immigration policy preferences, we argue that economic self-interest---specifically, the prospect of local job creation---can increase support for enforcement activities that citizens might otherwise oppose. We test this argument using a multi-method approach with a nationally representative sample of Americans. First, we conduct a pre-registered 2$\times$2 factorial vignette experiment that randomly varies (a) whether a proposed federal facility is framed as a ``detention center'' or ``processing facility'' and (b) whether economic benefits (jobs, local spending) are emphasized. Second, we deploy a conjoint experiment examining preferences over specific enforcement actions, varying target type, enforcement method, economic impact, and federal funding. Third, we embed treatments in a simulated social media environment to measure behavioral engagement. We complement these experiments with an observational analysis of public opinion in counties that have experienced ICE raids, examining whether support varies with local economic conditions. We find that [EXPECTED: emphasizing economic benefits significantly increases support for enforcement facilities, particularly among respondents in economically distressed areas and those with high economic anxiety, and that this effect operates even among those who are otherwise skeptical of immigration enforcement]. Our findings suggest that framing immigration enforcement as economic development may expand its coalition of support, with implications for how enforcement policies are marketed and contested in local communities.
\end{abstract}

\vspace{1cm}

\noindent \textbf{Keywords:} immigration enforcement, economic self-interest, framing effects, public opinion, conjoint experiment

\vspace{1cm}

\noindent \textbf{Word count:} [TBD]

\newpage

\section*{Research Questions}

\begin{enumerate}
    \item Does framing immigration enforcement facilities in terms of economic benefits (job creation) increase public support for such facilities?
    
    \item Does this economic framing differentially affect support among those who would otherwise oppose detention facilities (e.g., Democrats, those with pro-immigrant attitudes)?
    
    \item How do economic considerations trade off against other factors (target type, enforcement method) in shaping preferences over specific enforcement actions?
    
    \item Do local economic conditions moderate the effect of economic framing on enforcement support?
\end{enumerate}

\section*{Hypotheses}

\begin{description}
    \item[Economic Benefits Hypothesis:] Emphasizing job creation and economic benefits will increase support for immigration enforcement facilities, relative to descriptions that do not mention economic benefits.
    
    \item[Framing Interaction Hypothesis:] The positive effect of economic framing will be larger for facilities described as ``detention centers'' than for facilities described as ``processing facilities,'' as the economic benefits provide a counterweight to the negative valence of detention.
    
    \item[Economic Anxiety Moderation Hypothesis:] The effect of economic framing will be larger among respondents who report higher levels of economic anxiety and among those residing in areas with higher unemployment.
    
    \item[Partisan Heterogeneity Hypothesis:] Economic framing will have larger effects among Democrats and Independents than among Republicans, as Republicans already tend to support enforcement regardless of framing.
    
    \item[Conjoint Trade-off Hypothesis:] In the conjoint experiment, profiles emphasizing job creation will be more likely to be chosen, and this effect will be largest when paired with otherwise unpopular enforcement characteristics (e.g., targeting families).
\end{description}

\section*{Acknowledgments}

Thank you to the Maureen and Mike Mansfield Foundation and Japan Foundation for supporting important research on this project via the U.S.-Japan Network for the Future.

\section*{Data Availability}

We will make all data and code used to generate our results available at a figshare repository at the time of publication.

\section*{Ethics Statement}

We received ethics approval from [fill in] (protocol number: [fill in]).

\end{document}
